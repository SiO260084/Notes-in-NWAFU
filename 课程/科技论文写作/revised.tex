\documentclass[fontset=fandol]{ctexart}

\usepackage{geometry}

\geometry{left=1.5cm,right=1.5cm,top=1.5cm,bottom=1.5cm}

\begin{document}

\section{背景:确立PD与心血管疾病(特别是AMI)的潜在关联}

Over the last decades, accumulating evidence has suggested a potential link between periodontal disease and the development of cardiovascular conditions such as acute myocardial infarction (AMI), coronary heart disease (CHD), and other cardiovascular diseases. Several studies indicate that periodontal disease may be associated with an increased risk of AMI, particularly in men [4-5].

\section{定义:定义PD及其流行病学特征}

Periodontal disease is a chronic inflammatory condition characterized by the progressive destruction of the bone supporting the teeth. It is highly prevalent, affecting up to 75\% of adults over the age of 30 [6].

\section{机制:解释PD导致AMI的生物学通路}

Proposed mechanisms for the association suggest that periodontal pathogens, endotoxins, and inflammatory mediators—present both orally and systemically—may contribute to platelet aggregation and the formation of atherosclerotic plaques in coronary vessels [7-10]. Consequently, individuals with heightened inflammatory responses to periodontal infection may face an elevated risk of myocardial infarction [11].

\section{争议:明确指出当前研究领域的结论存在矛盾}

While some recent studies report an association between periodontal disease and coronary heart disease—even after adjusting for confounding variables—others have found no such link [2, 13-16]. This inconsistency renders the association controversial.

\section{方法学局限-样本:论证争议的第一个原因:样本问题(大小与偏倚)}

Methodological limitations may explain these conflicting results. For instance, studies supporting the association often involved small or selected samples [3, 4, 17]. Additionally, some studies included a high proportion of older men, which may introduce bias, given that AMI and CHD are common in this demographic.

\section{方法学局限-测量:论证争议的第二个原因:PD测量方法不统一}

Variability in the measurement of periodontal disease further complicates comparisons across studies. Many studies relied on attachment loss, periodontal indices, or self-reported measures [5], whereas few employed direct clinical assessments such as bacterial counts or serum antibody levels [18].

\section{本研究方案:基于上述局限,提出本研究将如何解决这些问题(样本、亚组、综合测量、严重度)}

To address these limitations, the present study will use a population-based sample of males aged 35–69 in Ontario to examine the association between periodontal disease and the incidence of a first myocardial infarction. Subgroup analyses by age will be conducted, and periodontal disease will be assessed comprehensively using blood cultures, IgG antibody tests, and standardized clinical examinations. The study will also investigate whether the severity of periodontal disease influences AMI risk.

 \section{假设与目标:清晰陈述研究假设和主要目标}

The primary hypothesis is that periodontal disease increases the risk of subsequent AMI among middle-aged men in Ontario with no prior history of coronary heart disease, stroke, transient cerebral ischemia, or cancer. The main objective is to examine the association between periodontal disease and the incidence of AMI in this population.
	
\end{document}