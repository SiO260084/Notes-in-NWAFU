\documentclass[a4paper,12pt,fontset=none,titlepage]{ctexart}
\usepackage{amsmath}
\usepackage{booktabs}
\usepackage{hyperref}
\usepackage{tabularx}

\setCJKmainfont{Noto Serif CJK SC}
\setCJKsansfont{Noto Sans CJK SC}
\setCJKmonofont{Noto Sans Mono CJK SC}

\hypersetup{colorlinks=true,linkcolor=blue}


\begin{document}
\author{石成玉}
\title{读基于超声波振动的肥箱抗粘减阻特性分析与装置设计}
\date{2025年9月24日}

\maketitle
\tableofcontents

\section{研究背景}

\begin{enumerate}
	\item \textbf{水果产业矛盾}:中国是水果生产大国,但不是水果生产强国。
	\item \textbf{没有充分利用中国的自然资源优势}:水果产业优质发展不仅需要充足的营养,还需要好的果园管理技术,尤其是\textbf{土壤养分管理}
	\item \textbf{土壤有机质含量低}:土壤有机质对果树优质生长发挥了重大的作用。施用有机肥可改善土壤有机质含量
\end{enumerate}

\section{问题和思考}

\subsection{为什么要开展有机肥施肥?}

\begin{enumerate}
	\item \textbf{水果产业矛盾:}中国是水果生产大国,但不是水果生产强国。
	\item \textbf{没有充分利用中国的自然资源优势:}水果产业优质发展不仅需要充足的营养,还需要好的果园管理技术,尤其是\textbf{土壤养分管理}
	\item \textbf{土壤有机质含量低:}土壤有机质对果树优质生长发挥了重大的作用。施用有机肥可改善土壤有机质含量
	\item 团队研制过一种果园施肥机,但是在施有机肥时会因为\textbf{肥料与箱底粘附}增加施肥阻力,甚至造成堵塞
	\item \textbf{减粘技术:}比如改变表面形貌减粘脱附技术、机械减粘脱附技术、加热减粘脱附技术等,都有其局限性
	\item \textbf{振动减粘:}低频振动能减轻触土部件的阻力;\textbf{高频低幅振动(超声波)}在金属加工中有较多应用,能\textbf{有效降低}加工时的阻力。
	\item \textbf{目的:}将超声波振动带到施肥机中
\end{enumerate}

\subsection{有机肥施肥装备存在问题是什么?}

施肥机施肥过程中易出现肥料与箱底粘附,导致排肥装置工作阻力增大,最终造成有机肥堵塞影响施肥机正常施肥等问题。

\begin{enumerate}
	\item \textbf{可遥控的履带式自走式果园开沟施肥机:}
	
    \begin{itemize}
		\item 未考虑肥料肥箱内流动性较差;
		\item 排肥装置容易发生堵塞
	\end{itemize}
	
	\item \textbf{分层施肥机:}
	
	\begin{itemize}
		\item 两侧土量过多导致前进阻力较大
	\end{itemize}
	
	\item \textbf{果园开沟施肥机}:
	
	\begin{itemize}
		\item 肥料粘附堵塞
	\end{itemize}
\end{enumerate}

\subsection{有机肥施肥存在的问题会带来哪些影响?}

肥料与肥箱底部黏附的问题会导致:

\begin{enumerate}
	\item \textbf{施肥阻力大:}

	\begin{itemize}
		\item 增加施肥时消耗的能量;
		\item 增加施肥机部分零件承受的载荷导致其过早失效;
	\end{itemize}
	
	\item \textbf{肥箱堵塞:}
	
	\begin{itemize}
		\item 使施肥中断,降低施肥效率;
		\item 频繁维修,增加工人劳动量;
		\item 施肥不均匀;
	\end{itemize}
\end{enumerate}

\section{论文试验中存在哪些问题}

\begin{enumerate}
	\item \textbf{有机肥颗粒与钢板的静摩擦系数的测量(Page.20,21)}
	
   有机肥颗粒是松散材料,在倾斜钢板的过程中,应该测量哪一时刻的倾斜角度?是否会出现有机肥顶部首先开始滑动的情况?

   \item \textbf{超声振动频率对有机肥摩擦角的影响(Page.23)}
   \begin{quote}
		\fbox{换能器端面的振幅降低,有机肥颗粒与钢板接触面积开始增大}
   \end{quote}
   
   振幅降低为什么会导致有机肥颗粒与钢板的接触面积增大?振幅改变是否影响有机肥颗粒与钢板的接触时间?
   
   \item \textbf{COMSOL中换能器与钢板的连接方式(Page.50)}
   
   实际中是否可以使用螺栓连接,以减少材料引起的振幅衰减。
\end{enumerate}

\section{智能控制方面如果你来做有哪些可以优化的地方?}

\begin{enumerate}
	\item 是否可以考虑使用不断变化的频率?
	\item 可以根据阻力的大小动态调节换能器的功率。
	\item 可以在出肥孔增加流量传感器以保证出肥速率稳定,使施肥均匀。
	\item 可以增加堵塞处理机构,使得堵塞时能自行恢复。
\end{enumerate}

\subsection{其他问题}

\subsubsection{超声振动施加方式·纵向振动施加方式}

\begin{table}[!h]
	\centering
	\begin{tabular}{cc}
		\toprule
		变量 & \\
		\midrule
		有机肥在钢板上的恒定滑动速度 & $V_0$ \\
		摩擦副的振幅 & $\alpha$ \\
		角速度 & $\omega$ \\
		\bottomrule
	\end{tabular}
\end{table}

振动函数

\[V_{(t)}=\alpha \omega \sin ( \omega t) \tag{2-1}\label{振动函数}\]

整个循环阶段的平均摩擦力

\[F = \frac{F_0}{T}(4t_s) \tag{2-2}\label{average_force}\]

其中

\[T=\frac{2\pi}{\omega} \tag{2-3}\label{周期}\]

又公式$\eqref{average_force}$和$\eqref{周期}$得:

\[t_s=\frac{FT}{4F_0}=\frac{F}{4F_0}\times\frac{2\pi}{\omega}=\frac{F\pi}{2F_0\omega} \tag{2-4}\]

\end{document}